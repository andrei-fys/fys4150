\documentclass[10pt]{article}
\usepackage{mathtools}
\usepackage{amsmath}
\usepackage{tabularx}
\usepackage{graphicx}
\usepackage{flexisym}
\usepackage{listings}
\usepackage{xcolor}
\usepackage{hyperref}
\usepackage{amsthm}
\usepackage{subcaption}
\newtheorem*{theorem}{Theorem}
\begin{document}
\setlength\parindent{1pt}
\title{Project 2}
\author{Andrei Kukharenka and Anna Gribkovskaya \\  
FYS 4150 
}

\maketitle
\begin{abstract}
In this work we have solved the Schr\"{o}dinger's equation for electrons in three dimensional harmonic oscillator well. The solution obtained by approximating the derivatives with three point formula. We have used the Jacobi method for the matrix diagonalization to find eigenvalues and eigenvectors. The results have been obtained for one and two electron cases. For two electrons we have studied both cases with and without interaction. All results found were in agreement with those obtained analytically. The method is rather time consuming, but stable and provide a good numerical precision. 
\end{abstract}
\clearpage 


\section{Introduction}
This project is devoted to three-dimensional Schr\"{o}dinger's equation for electrons a harmonic oscillator potential. We solved this equation for one and for two electrons (both with and without Coloumb interaction between particles). We also have considered different strengths of harmonic oscillator for both cases. \\*
Particles inside the harmonic oscillator potential is quite common problem in physics. Harmonic oscillator confine particles in small areas in the space and form so-called quantum dots that a widely used in experiments and also in industry (for example for the displays). \\*
In this project we have solved the three-dimensional Schr\"{o}dinger's equation numerically by discretizing it with three point formula. After this the problem becomes the eigenvalue problem for three diagonal matrix that we have solved with Jacobi rotations method for eigenvalues. This method is one of the most straight forward methods for such problem. However as we have seen not the most efficient and quite time consuming. \\*
Our work has the following structure:\\*
In the Problem formulation and mathematical method \label{Part1} part we discuss the problem and the approximation of the derivatives together with the numerical method algorithm. \\*
In the Results and Discussion part we   have presented all the data, discuss the method and the obtained results. \\*
In Conclusion we made a brief overview of what have been done in the project. 

\newpage
\section{Problem formulation and mathematical method}\label{Part1}
In this project we consider have solved Schr\"{o}dinger's equation for electrons confined in a small area. First we took a one electron case and after this moved to a two-electrons repelling due to Coloubm interaction. 

\subsection{One electron case}
We are first interested in the solution of the radial part of Schr\"{o}dinger's equation for one electron. 

\begin{equation*}
  -\frac{\hbar^2}{2 m} \left ( \frac{1}{r^2} \frac{d}{dr} r^2
  \frac{d}{dr} - \frac{l (l + 1)}{r^2} \right )R(r) 
     + V(r) R(r) = E R(r).
\end{equation*}
The potential $V(r)$  in the equation is a harmonic oscillator potential $(1/2)kr^2$ and $E$ is the energy of the harmonic oscillator. 

The boundary conditions are $u(0)=0$ and $u(\infty)=0$.

Assuming spherical symmetry and considering the orbital momentum $l=0$ we made some simple transformations and variable substitutions. Now the equation reads as


\begin{equation}
-\frac{d^{2}}{d\rho ^{2}}u(\rho )+\rho ^{2}u(\rho )=\lambda u(\rho )
\end{equation}

Using the standard expression for $u^{\prime \prime }$we obtain 
\begin{equation}
u^{\prime \prime }=\frac{u(\rho +h)-2u(\rho )+u(\rho -h)}{h^{2}}+O(h^{2})
\end{equation}

where $h$ is our step, given by

\begin{equation}
h=\frac{\rho _{\mathrm{max}}-\rho _{\mathrm{min}}}{n_{\mathrm{step}}}
\end{equation}

where $\rho _{\mathrm{max}}$ and $\rho _{\mathrm{min}}$ are given by boundary conditions. We assume $\rho _{\mathrm{min}}=0$, and define $\rho_{\mathrm{max}}$ as a most suitable for the applied algorithm.

Value of $\rho $ is given by

\begin{equation}
\rho _{i}=\rho _{\mathrm{min}}+ih\hspace{1cm}i=0,1,2,\dots ,n_{\mathrm{step}}
\end{equation}
we can rewrite the Schr\"{o}dinger equation for $\rho _{i}$ as

\begin{equation}
-\frac{u_{i+1}-2u_{i}+u_{i-1}}{h^{2}}+V_{i}u_{i}=\lambda u_{i}
\end{equation}
where $V_{i}=\rho _{i}^{2}$ is the harmonic oscillator potential. 

The diagonal matrix element are defined as

\begin{equation}
d_{i}=\frac{2}{h^{2}}+V_{i}
\end{equation}
the non-diagonal matrix element are
defined as 

\begin{equation}
e_{i}=-\frac{1}{h^{2}}
\end{equation}

In this case the Schr\"{o}dinger equation takes the following form

\begin{equation}
d_{i}u_{i}+e_{i-1}u_{i-1}+e_{i+1}u_{i+1}=\lambda u_{i}
\end{equation}
where $u_{i}$ is unknown. We can write the last equation as a matrix $%
\mathbf{A}$ eigenvalue problem

\[
\left( 
\begin{array}{ccccccc}
d_{1} & e_{1} & 0 & 0 & \dots  & 0 & 0 \\ 
e_{1} & d_{2} & e_{2} & 0 & \dots  & 0 & 0 \\ 
0 & e_{2} & d_{3} & e_{3} & 0 & \dots  & 0 \\ 
\dots  & \dots  & \dots  & \dots  & \dots  & \dots  & \dots  \\ 
0 & \dots  & \dots  & \dots  & \dots  & d_{n_{\mathrm{step}}-2} & e_{n_{%
		\mathrm{step}}-1} \\ 
0 & \dots  & \dots  & \dots  & \dots  & e_{n_{\mathrm{step}}-1} & d_{n_{%
		\mathrm{step}}-1}%
\end{array}%
\right) \left( 
\begin{array}{c}
u_{1} \\ 
u_{2} \\ 
\dots  \\ 
\dots  \\ 
\dots  \\ 
u_{n_{\mathrm{step}}-1}%
\end{array}%
\right) =\lambda \left( 
\begin{array}{c}
u_{1} \\ 
u_{2} \\ 
\dots  \\ 
\dots  \\ 
\dots  \\ 
u_{n_{\mathrm{step}}-1}%
\end{array}%
\right) 
\]%
or if we wish to be more detailed, we can write the tridiagonal matrix $%
\mathbf{A}$ as

\[
\left( 
\begin{array}{ccccccc}
\frac{2}{h^{2}}+V_{1} & -\frac{1}{h^{2}} & 0 & 0 & \dots  & 0 & 0 \\ 
-\frac{1}{h^{2}} & \frac{2}{h^{2}}+V_{2} & -\frac{1}{h^{2}} & 0 & \dots  & 0
& 0 \\ 
0 & -\frac{1}{h^{2}} & \frac{2}{h^{2}}+V_{3} & -\frac{1}{h^{2}} & 0 & \dots 
& 0 \\ 
\dots  & \dots  & \dots  & \dots  & \dots  & \dots  & \dots  \\ 
0 & \dots  & \dots  & \dots  & \dots  & \frac{2}{h^{2}}+V_{n_{\mathrm{step}%
	}-2} & -\frac{1}{h^{2}} \\ 
0 & \dots  & \dots  & \dots  & \dots  & -\frac{1}{h^{2}} & \frac{2}{h^{2}}%
+V_{n_{\mathrm{step}}-1}%
\end{array}%
\right) 
\]

The boundary conditions in this case are for $i=n_{\mathrm{step}}$ and for $%
i=0$. The solution is zero in both cases.\\*
\subsection{Jacobi algorithm for one electron case}
According to diagonalization theorem :
\begin{theorem}

If we have a matrix $ A $ of $ n \times n$ dimensions this matrix diagonable if and only if $ A $ has $ n $ linearly independent eigenvectors of $ A $. Suppose $ v_{1} \dots v_{n} $ is a linearly independent set of eigenvectors of $ A $ with corresponding eigenvalues $ \lambda_{1} \dots \lambda_{n} $. Then the matrix $ P = (v_{1} |\dots |v_{n}) $ is invertible so that we have the following equality:
\begin{equation}
P^{-1}AP=Diag(\lambda_{1},\dots,\lambda_{n});
\end{equation}
\end{theorem}
In our project we assumed that our matrix is diagonable and used Jacobi diagonalization method to find the corresponding eigenvector and eigenvalues. Below we present the Jacobi algorithm for the eigenvalue problem. 
\\*
Baby-steps for Jacobi algorithm:

\begin{enumerate}
\item One should choose a parameter $\varepsilon $ in order to define tolerance. The parameter should be as close to zero as possible, as soon as we cannot get exact zero value.

\item After this one should find the largest non-diagonal matrix element $\left\vert
a_{kl}\right\vert =\max\nolimits_{i\neq j}\left\vert a_{ij}\right\vert $and
compare it to $\varepsilon $. The program should run until $\varepsilon $ is smaller then any of the off-diagonal elements.

\item As soon as our element is larger then epsilon we need to perform a so-called rotation of the matrix. In order to do this we first define the angle of rotation and construct the rotation matrix (Jacobi matrix). The angle $\theta $ of rotation should be chosen so the largest off-diagonal element become zero.
We did not compute exactly $\theta $, but use $%
\tan \theta =t=s/c$, with $s=\sin \theta $ and $c=\cos \theta $ and $\cot
2\theta =\tau $. 
In our case $\tau $ can be defined as follows:

\begin{equation}
\tau =\frac{a_{ll}-a_{kk}}{2a_{kl}}
\end{equation}%

We define the angle $\theta $ so that the non-diagonal matrix
elements of the transformed matrix $a_{kl}$ become non-zero and we obtain
the quadratic equation (using $\cot 2\theta =1/2(\cot \theta -\tan \theta )$

\begin{equation}
t^{2}+2\tau t-1=0
\end{equation}
resulting in

\begin{equation}
t=-\tau \pm \sqrt{1+\tau ^{2}}
\end{equation}
This equation for $ t $ may lead to some problems when we use it in our program. One can mention that for large $ \tau $ the nominator is close to zero, wich is not good when it come to machine representation of the numbers. We also have to take into account that we want to avoid rotation for more the $ |\frac{\pi}{4}|$. In order to avoid the roundoff errors in the computer and to secure rotation only for small angels $ \theta $ we will rewrite the formula in a following way.
\begin{gather*}
t=\frac{-1}{\tau+\sqrt{1+\tau^{2}}}\,\ \text{for} \  \tau>0 \\
t=\frac{-1}{-\tau+\sqrt{1+\tau^{2}}}\,\ \text{for} \ \tau<0
\end{gather*}


and $c$ and $s$ are given by

\begin{equation}
c=\frac{1}{\sqrt{1+t^{2}}}
\end{equation}
and $s=tc$. 

\item 
After we get $s$ and $c$ we need to calculate new matrix elements



\begin{eqnarray*}
\acute{a}_{kk} &=&c^{2}a_{kk}-2csa_{kl}+s^{2}a_{ll} \\
\acute{a}_{ll} &=&s^{2}a_{kk}+2csa_{kl}+c^{2}a_{ll} \\
\acute{a}_{ik} &=&ca_{ik}-sa_{il} \\
\acute{a}_{il} &=&ca_{il}+sa_{ik} \\
\acute{a}_{ki} &=&\acute{a}_{ik} \\
\acute{a}_{li} &=&\acute{a}_{il}
\end{eqnarray*}

\item We run the algorithm, until $\max \left\vert a_{ij}\right\vert \leq
\varepsilon ,i\neq j.$ 
\end{enumerate}

\subsection{Two electron case}
For the two electron case  we will use the same algorithm, but with some
sufficient changes. Here we need to consider two cases with and without
Coulomb interaction.

With no repulsive Coulomb interaction, we have the following Schr\"{o}dinger
equation

\begin{equation}
	\left( -\frac{\hbar ^{2}}{2m}\frac{d^{2}}{dr_{1}^{2}}-\frac{\hbar ^{2}}{2m}%
	\frac{d^{2}}{dr_{2}^{2}}+\frac{1}{2}kr_{1}^{2}+\frac{1}{2}kr_{2}^{2}\right)
	u(r_{1},r_{2})=E^{(2)}u(r_{1},r_{2})
\end{equation}


	A two-electron wave function $u(r_{1},r_{2})$ for the case with no
	interaction can be written out as the product of two single-electron wave
	functions.
	
	Now we need to find wave function and energy for the case of two electron
	with Coulomb interaction. After substitution of the variables and
	introduction of some additional parameters we have the following equation:
	
	\begin{equation}
		-\frac{d^{2}}{d\rho ^{2}}\psi (\rho )+\omega _{r}^{2}\rho ^{2}\psi (\rho )+%
		\frac{1}{\rho }=\lambda \psi (\rho )
	\end{equation}
	
	with 
	\begin{equation}
		\omega _{r}^{2}=\frac{1}{4}\frac{mk}{\hbar ^{2}}\alpha ^{4}
	\end{equation}
	
	$\omega _{r}$ here is a parameter which reflects the strength of the
	oscillator potential. Other parameter are:%
	\begin{equation}
		\alpha =\frac{\hbar ^{2}}{m\beta e^{2}}
	\end{equation}%
	\begin{equation}
		\lambda =\frac{m\alpha ^{2}}{\hbar ^{2}}E
	\end{equation}
	
	and $\beta e^{2}=1.44$ eVnm.
	
We will solve this eigenvalue problem the same way as we did it for one
electron case. The only difference now is the potential. We used $\omega _{r}^{2}\rho ^{2} $ as a new potential for non-interacting case and  $\omega _{r}^{2}\rho ^{2}+\frac{1}{\rho }$ for interaction case. This lead to some interesting effects we will discuss in the next par of the report.

\newpage
\section{Results and discussion}
In Table \ref{tab:one} we presented the eigenvalues for the one electron case for different matrix sizes. We know form analytical solution that the correct eigenvalues are $ \lambda_{1}=3 $, $ \lambda_{2}=7 $ and $ \lambda_{3}=11 $. As we can see we have results close to these even for rather small matrix sizes. For matrix size $ 400 \times 400 $ we have a relative error less then $ 1\% $ for the first $ \lambda $. That is quite good result. However, the number of rotations increase dramatically when we increase the matrix size. This make Jacobi algorithm very time consuming and not applicable for large matrices.
\begin{table}
  \caption{Three first eigenvalues for one electron in harmonic oscillator potential obtained using Jacobi algorithm. Matrix size $200 \times 200$}
  \label{tab:one}
  \begin{center}
    \begin{tabular}{c|c|c|c|c}
    \hline
		$N$ & $Number of rotations$ & $\lambda_1$ & $\lambda_2$ & $\lambda_3$ \\
        \hline
	$	50 $  & $ 4036  $ & $2.77921$ & $6.65469$ & $10.5485$ \\
	$	100$  & $ 16521 $ & $2.88836$ & $6.82895$ & $10.7813$ \\
	$	200$  & $ 66745 $ & $2.94388$ & $6.91491$ & $10.8925$ \\
	$	300$  & $ 150755$ & $2.96252$ & $6.94338$ & $10.9288$ \\
	$	400$  & $ 269516$ & $2.97186$ & $6.95757$ & $10.9468$ \\

	\end{tabular}
  \end{center}
\end{table}

\begin{table}
  \caption{Three first eigenvalues for two electrons in harmonic oscillator potential for different values of $\omega$. Results are obtained using Jacobi algorithm. Matrix size $200 \times 200$}
  \label{tab:two}
	\begin{center}
    \begin{tabular}{c|c|c|c|c}
    \hline
		$\omega$ & $Number\ of\ rotations$ & $\lambda_1$ & $\lambda_2$ & $\lambda_3$ \\
        \hline
		$0.01$ & $59717$ & $0.105776$ & $0.141516$  & $0.178049$ \\ 
		$0.5$  & $64207$ & $2.25271 $ & $4.17817 $  & $6.13601$ \\ 
		$1  $  & $64662$ & $4.11774 $ & $8.01741 $  & $11.966$ \\
		$5  $  & $66625$ & $17.8299 $ & $37.6929 $  & $57.7022$ \\

	\end{tabular}
  \end{center}
\end{table}




\begin{figure}
  \begin{center}
    \includegraphics[scale=0.7]{one_electron}
    \caption {Probability function ($|\Psi^{2|$}) dependence from radial coordinate $\rho $ for one electron in harmonic oscillator potential.}
    \label{fig:one_electron}
  \end{center}
\end{figure}


\begin{figure}[h!] 
  \begin{subfigure}[b]{0.6\linewidth}
    \centering
    \includegraphics[width=1.1\linewidth]{two_001} 
    \caption{two_001} 
    \label{fig1:a} 
    \vspace{1ex}
  \end{subfigure}%% 
  \begin{subfigure}[b]{0.6\linewidth}
    \centering
    \includegraphics[width=1.1\linewidth]{two_05} 
    \caption{two_05} 
    \label{fig1:b} 
    \vspace{1ex}
  \end{subfigure} 
  \begin{subfigure}[b]{0.6\linewidth}
    \centering
    \includegraphics[width=1.1\linewidth]{two_1} 
    \caption{two_1} 
    \label{fig1:c} 
  \end{subfigure}%%
  \begin{subfigure}[b]{0.6\linewidth}
    \centering
    \includegraphics[width=1.1\linewidth]{two_5} 
    \caption{two_5} 
    \label{fig1:d} 
  \end{subfigure} 
  \caption{ Total energy $E_{tot}$, maximal force, $F_{max}$ and pressure $P$ of $SiO_2$ calculated with different values of energy cut-off $E_{cut}$.}
  \label{fig1} 
\end{figure}



\newpage

\begin{figure}[h!] 
  \begin{subfigure}[b]{0.6\linewidth}
    \centering
    \includegraphics[width=1.1\linewidth]{two_nint_001} 
    \caption{two_nint_001} 
    \label{fig1:a} 
    \vspace{1ex}
  \end{subfigure}%% 
  \begin{subfigure}[b]{0.6\linewidth}
    \centering
    \includegraphics[width=1.1\linewidth]{two_nint_05} 
    \caption{two_nint_05} 
    \label{fig1:b} 
    \vspace{1ex}
  \end{subfigure} 
  \begin{subfigure}[b]{0.6\linewidth}
    \centering
    \includegraphics[width=1.1\linewidth]{two_nint_1} 
    \caption{two_nint_1} 
    \label{fig1:c} 
  \end{subfigure}%%
  \begin{subfigure}[b]{0.6\linewidth}
    \centering
    \includegraphics[width=1.1\linewidth]{two_nint_5} 
    \caption{two_nint_5} 
    \label{fig1:d} 
  \end{subfigure} 
  \caption{ Total energy $E_{tot}$, maximal force, $F_{max}$ and pressure $P$ of $SiO_2$ calculated with different values of energy cut-off $E_{cut}$.}
  \label{fig1} 
\end{figure}


\newpage

\section{Conclusion and further research}
In this project we studied several problems. The one electron in harmonic oscillator potential, which is analytically solvable, and two electron in the harmonic oscillator potential. The latter we have studied for different strengths of harmonic oscillator. We also considered case with and without Coulomb interaction.
For the one electron case both eigenvalues and eigenvectors are in a good agreement with analytically obtained results.

\newpage
\begin{thebibliography}{2}
\bibitem{one} 
Morten Hjorth-Jensen. 
\textit{Computational Physics
}. 
Lecture Notes Fall 2015, August 2015.


\bibitem{two} 
W. Press, B. Flannery, S. Teukolsky, W. Vetterling 
\textit{Numerical Recipes in C++, The art of scientific Computing}. 
Cambridge University Press, 1999.
 
\end{thebibliography}

\end{document}
