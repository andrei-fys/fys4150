\documentclass[10pt]{article}
\usepackage{mathtools}
\usepackage{amsmath}
\usepackage{tabularx}
\usepackage{graphicx}
\usepackage{flexisym}
\usepackage{listings}
\usepackage[most]{tcolorbox}
\usepackage{xcolor}
\usepackage{hyperref}
\usepackage{amsthm}
\usepackage{subcaption}
\usepackage[a4paper,top=3cm,bottom=2cm,left=3cm,right=3cm,marginparwidth=1.75cm]{geometry}
\newtcblisting[auto counter]{pseudolisting}[2][]{sharp corners,
    fonttitle=\bfseries, colframe=gray, listing only,
    listing options={basicstyle=\ttfamily,language=c++},
    title=Listing \thetcbcounter: #2, #1}
\newtheorem*{theorem}{Theorem}

\begin{document}
\setlength\parindent{1pt}
\title{Quantum Monte Carlo of confined electrons }
\author{Anna Gribkovskaya and Andrei Kukharenka \\  
FYS 4150 
}

\maketitle
\begin{abstract}
Aim of this work is to determine the ground state energy of the quantum system using Variational Monte Carlo (VMC) method. The considered system is two electrons confined in quantum dot. Quantum dot is approximated with a three-dimensional harmonic oscillator potential. Apart of ground state energy the relative distance between two electrons and expectation values of the kinetic and potential energies were evaluated. Obtained results are compared with those obtained in second project\cite{proj2} and analytical results from Taut \cite{three}. 
\end{abstract}
\clearpage 


\section{Introduction}
System under consideration is a three-dimensional harmonic oscillator potential with two electrons. Such potential traps electrons inside and prevent them to move apart. Such systems are called quantum dots and have a large application in science \cite{four}, industry\cite{five} and medicine \cite{med}. \\
We already study such system in project two \cite{proj2} emploing brute-force approach -- using the eigenvalue solver, so we are able to compare results from different methods. The advantage of VMC method is that we can use Cartesian coordinates and do not need to transform the equations. \\ 
VMC is a method for solving Schr\"{o}dinger's equation by using Metropolis sampling to simulate Markov processes. This method may seem not as straightforward as the direct approach of solving eigenvalue problem, but it is much closer to the "nature" of quantum mechanics. The quantum world is a world of probabilities and stochastic methods are much more appropriate to study the system, even though we are limited to find only one most probable state \cite{one}. \\
All equations presented in the report are in atomic units, which means that all constants, such as speed of light or Planck's constant, are set to 1 ($\hbar=c=e=m_e=1$).
The report has the following structure:\\*
We begin with discussing the nature of the problem in \ref{Part1}.
In results and discussion section \ref{results} we presented all the data, plots and analysis for obtained results. 
In conclusion \ref{conc} we made a brief overview of obtained results and discuss possibilities for further research. 

\newpage
\section{Variational Monte-Carlo method for quantum dot study}\label{Part1}
\subsection{Motivation for variational Monte-Carlo}
Variational methods are widely used in quantum physics. The basis for these methods is variational principle and a key idea is to use some trial wave function which depend on many parameters. Varying parameters one can minimize the expectation value of the energy. Many other ways of solving Schr\"{o}dinger equation are based on this idea, for example the Hartree, Hartree–Fock and even density functional methods.\\
However variational principle based methods has a drawback -- one can find only ground state properties. Reader interested in detailed description of
variational Monte Carlo method can find it in lecture-notes of Hjorth-Jensen \cite{one}. Our goal here is to present basic ideas enough to set up algorithm.
In this project we consider simulations for the Markov chains (or random walkers) with sampling employed by Metropolis algorithm. The essence of such process is that it depend only on a previous move and do not have "memory" about the all previous.
To understand how to set algorithm we introduce several quantities. Local energy is expressed as:

\begin{equation*}
E_L = \frac{1}{\Psi_T(R)} \hat{H} \Psi_T(R)
\end{equation*}

which represents energy of system after one move. In our case of two electrons it quantity depends on distance between them($R$).
$\Psi_T(R)$ is trial wave function we construct gathering all known physical properties of a system.
Expectation value of energy becomes:
\begin{equation*}
E = \int_{}^{} \rho(R)E_L dR
\end{equation*}
where probability density distribution $\rho(R)$ defined by:
\begin{equation*}
\rho(R) = \frac{|\Psi_T(R)|^2}{\int |\Psi_T(R)|^2\rho(R)E_L dR}
\end{equation*}
Trial wave function is varied to get energy minimum. Energy can be obtained as mean of local energy.

Below we describe how to set up variational Monte Carlo method for system of two electrons and give analytical expression for trial wave functions.

\subsection{Quantum dots study}
Random walker in our case jumps in real space. To accept or reject a new move we employ Metropolis algorithm to check transition probability.
In developed C++ program we move two electrons per one Monte Carlo experiment. To obtain an optimal step for moving electrons we run a electrons jump estimation procedure before actual calculations. Namely with lower number of Monte Carlo samples we run Metropolis sampling starting with guessed value of step and save step value when we reach roughly 50 $\pm $ 1 \% of accepted/rejected moves ratio. We do ten such experiments and then take mean value of step for set of variational parameters(will be discussed bellow).
\begin{pseudolisting}{optimal step finding}
for (int j=0; j < 10; j++) {
 for ( int i=0; i < 500; i++ ){
  Metropolis(...);
   if (abs(MC_rejected_prosent - 50.0) < 1.0 ){
    break;
   }
   h = h0 + step*i;
 }
 mean_h += h;
}
mean_h = mean_h/10.0;
\end{pseudolisting}


\subsubsection{Algorithm implementation}
The algorithm itself is rather simple and can be described in a following way:
 \begin{enumerate}
\item Initialize all values to be updated using the algorithm. In our case we update local energy, kinetic energy, potential energy and distance between the particles.
\item Start Monte Carlo simulations
  \begin{enumerate}
  \item Using a random number generator (RNG), generate a new position $R_{new} = R + r*h$. Here $R$ is old position and value $h$ is an optimal step. We have already discussed above how to find this value. Value $r$ is a random number in range $[-1,1]$, so our random walker can "jump" in both directions. 
  \item Here we use the Metropolis sampling to decide whether to accept or reject the move towards new position. In order to do this we need transition probability $W$. 
  \begin{equation}
  W=\frac{P(R_{new})}{P(R)}
  \end{equation}
  here $P(R)$ is probability given by $|\Psi_T|^2$. So, transition probability $W$ will obviously be different for different trial wave functions.\\
  If $W \geq 1$ the new position is accepted. If not, we need use RNG again and generate a number $s$ in range $[0,1]$. \\
  If $W \geq s$ the new position is accepted, else random walker stay at the same place.
    \item If step was accepted, we update the position by setting $R=R_{new}$.
    \item After Metropolis test done and decision have been made, we update all values we need (local energy, kinetic energy, potential energy and distance between the particles).
\end{enumerate}
\end{enumerate}

\subsubsection{Theoretical background}

The Hamiltonian of system under consideration is given by
\begin{equation}
  \label{eq:finalH}
  \hat{H}=\sum_{i=1}^{N} \left(  -\frac{1}{2} \nabla_i^2 + \frac{1}{2} \omega^2r_i^2  \right)+\sum_{i<j}\frac{1}{r_{ij}},
\end{equation}
In our case $N=2$. 
Any introductory textbook on quantum mechanics, for example \cite{Liboff}, has an expression for the eigenfunctions of the harmonic oscillator in form of Hermitian polynomials.  Our system is a three dimensional harmonic oscillator, which in Cartesian coordinates can be separated into three independent harmonic oscillators. This gives us energy in a form:
\begin{equation*}
E=\omega(n_x+n_y+n_z+3/2) 
  \end{equation*}
  The eigenfunctions in a form of Hermitian polynomials are
\begin{equation*}
  \phi_{n_x,n_y,n_z}(x,y,z) = A H_{n_x}(\sqrt{\omega}x)H_{n_y}(\sqrt{\omega}y)H_{n_z}(\sqrt{\omega}z)\exp{(-\omega(x^2+y^2+z^2)/2}.
  \end{equation*}
 here $A$ is a normalization constant. In the algorithm discussed above we compute the transition probability as a ratio, so we normally do not care about the constants. However this can be calculated using the Slater determinant. 
 Using the formulas above we can get a lowest-lying energy for the two-electron system which is $3\omega$. We will use this value as a unit test of our program.

As all other variational methods the VMC uses trial functions. We will consider two different trial functions. 
First trial wave function have the following form:
\[
   \Psi_{T1}(\mathbf{r}_1,\mathbf{r}_2) = C\exp{\left(-\alpha\omega(r_1^2+r_2^2)/2\right)},
\]
here  $\alpha$ is variational parameter, $\omega$ is strenth of harmonic oscillator potential, $r_1, r_2$ are coordinates of electrons. Total spin of the system in this case is zero because of Pauli exclusion principle.

Inserting this trial wave function into Schr\"{o}dinger's equation and solving it for eigenvalues we get the the local energy in form: 
\[ 
E_{L1} = \frac{1}{2}\omega^2\left( r_1^2+r_2^2\right)\left(1-\alpha^2\right) +3\alpha\omega.
\]

If we consider interaction between particles trial wave function should include terms corresponding external potential as well as potential energy from internal interaction. Latter can be introduced by a Jastrow factor.
The second trial wave function in this case has the following form:
\[
    \Psi_{T2}(\mathbf{r}_1,\mathbf{r}_2) =
    C\exp{\left(-\alpha\omega(r_1^2+r_2^2)/2\right)}
    \exp{\left(\frac{r_{12}}{2(1+\beta r_{12})}\right)},
\]
hare $\beta$ is variational parameter.
Do the same as for first wave function and get the expression for local energy for second wave function
\[ 
E_{L2} = \frac{1}{2}\omega^2\left( r_1^2+r_2^2\right)\left(1-\alpha^2\right) +3\alpha\omega+\frac{1}{r_{12}}+\frac{1}{2(1+\beta r_{12})^2} 
\left\{\alpha\omega r_{12}-\frac{1}{2(1+\beta r_{12})^2}-\frac{2}{r_{12}}+\frac{2\beta}{1+\beta r_{12}}\right\}
\]
\subsubsection{Virial theorem}
This theorem states a relation between the potential and kinetic energies for stable systems. For the harmonic oscillator this result in 

\begin{equation*}
  \langle T \rangle= \langle V \rangle.
  \end{equation*}
here $\langle T \rangle$ is expectation value of the total kinetic energy and   $\langle V \rangle$ expectation value of the total potential energy.

\newpage
\section{Results and discussion}\label{results}

In this project we start with non interacting system. We use first trial wave function to determine the optimal value of variational parameter $\alpha$ and also to check the result against the one we got analytically in the section above. For the non-interacting case, using the first trial wave function our program reproduce the same result as obtained analytically, see Table \ref{tab:one}. \\
We also studied accuracy of method against the number of Monte Carlo cycles and found it to be rather accurate. We use $10^6$ Monte Carlo cycles for all calculations, but even $10^4$ can provide good result (at least such computation can be used as a rough one for estimating the optimal step of random walker jump). 
From Fig. \ref{fig1} one can see that the optimal variational parameter $\alpha$ for this case is 1. In Table \ref{tab:one} we presented the mean distance at the energy minimum for the non-interaction case. As one can see from the table mean distance between two electrons become bigger as the harmonic oscillator strength $\omega$ decreases. This result agree with that obtained in project two \cite{proj2}. The stronger the harmonic oscillator is the closer it keeps the particles together even the repulsion is inversely proportional to the distance. As expected, the harmonic oscillator potential dominates the Column potential for large $\omega$. 

\begin{table}[h!]
  \caption{Expectation energy, variance for energy and relative distance between two non-interacting electrons for different values of $\omega$. Variational parameter $\alpha =1$}
  \label{tab:one}
  \begin{center}
    \begin{tabular}{c|c|c|c}
    \hline
		$\omega$& Expectation Energy & $\sigma^2_{energy}$   & Expectation relative distance  \\
    \hline
	$	1 $  &  $3.0$ & $0$ & $1.5964$  \\
	$	0.5$   &$1.5$ & $0$& $2.2565$   \\
	$	0.01$   & $0.03$ &$2\times  10^{-20}$ &  $15.977$   \\
	\end{tabular}
  \end{center}
\end{table}



\begin{figure}[h!] 
  \begin{subfigure}[b]{0.55\linewidth}
    \centering
    \includegraphics[width=0.9\linewidth]{energy_on_alpha_001} 
    \caption{The $\omega$ is 0.01} 
    \label{fig1:a} 
    \vspace{1ex}
  \end{subfigure}%% 
  \begin{subfigure}[b]{0.55\linewidth}
    \centering
    \includegraphics[width=0.9\linewidth]{energy_on_alpha_05} 
    \caption{The $\omega$ is 0.5} 
    \label{fig1:b} 
    \vspace{1ex}
  \end{subfigure} 
  \begin{subfigure}[b]{0.55\linewidth}
    \centering
    \includegraphics[width=0.9\linewidth]{energy_on_alpha_1} 
    \caption{The $\omega$ is 1} 
    \label{fig1:c} 
  \end{subfigure}%%
  \begin{subfigure}[b]{0.55\linewidth}
    \centering
    \includegraphics[width=0.9\linewidth]{variance_on_alpha} 
    \caption{The $\omega$ is 1} 
    \label{fig:d} 
  \end{subfigure} 
  \caption{ Energy (a) -- (c) and energy variance (d) as a function of variational parameter $\alpha$.}
  \label{fig1} 
\end{figure}

\begin{figure}[h!] 
  \begin{subfigure}[b]{0.55\linewidth}
    \centering
    \includegraphics[width=0.9\linewidth]{Ratio_non_ineracting} 
    \caption{Non-interacting case} 
    \label{fig2:a} 
    \vspace{1ex}
  \end{subfigure}%% 
  \begin{subfigure}[b]{0.55\linewidth}
    \centering
    \includegraphics[width=0.9\linewidth]{ratio_on_omega_ineraction} 
    \caption{Interacting case} 
    \label{fig2:b} 
    \vspace{1ex}
  \end{subfigure} 
  \caption{Virial theorem verification. $\langle T \rangle$ is expectation value of the total kinetic energy and   $\langle V \rangle$ expectation value of the total potential energy.} 
  \label{fig2} 
\end{figure}

The second trial wave function depend on two parameters $\alpha$ and $\beta$. In this case we need to vary both parameters and find the minimal energy. All results are presented in Table \ref{tab:two}. As one can mention the expectation value of the energy increases in this case and the larger $\omega$ is the larger the difference in the expectation energy is. This is as well quite obvious if we remember that we have added a term inversely proportional to the relative distance and as we already mention above relative distance decreases as $\omega$ increases. Thus for large $\omega$ this addition becomes larger.
 
\begin{table}[h!]
  \caption{Relative distance between two interacting electrons and minimal energy for different values of $\omega$. $\alpha$ and $\beta$ are variational parameters. $\sigma^2_{energy}$ is variance for energy.}
  \label{tab:two}
  \begin{center}
    \begin{tabular}{c|c|c|c|c|c}
    \hline
		$\omega$ & $\alpha$& $\beta$ & Expectation Energy& $\sigma^2_{energy}$ & Expectation relative distance \\
    \hline
	$	1 $  & $0.9845$& $0.355$ & $3.73094$ & $1.63921 \times 10^{-9}$ & $2.46374$  \\
	$	0.5$  & $0.93$ & $0.52$ & $2.00877$ & $1.19493 \times 10^{-8}$ & $2.53494$   \\
	$	0.01$  & $0.7$ & $0.08$ & $0.03196$ & $3.75944 \times 10^{-11}$ & $27.5623$   \\
	\end{tabular}
  \end{center}
\end{table}


On Fig. \ref{fig2} we presented result for studying the virial theorem. In non-interacting case the ratio between expectation  $\langle T \rangle$  and   $\langle V \rangle$ is close to 1. If we consider the Column interaction the ratio between  $\langle T \rangle$  and   $\langle V \rangle$ begin to vary with $\omega$. On figure \ref{fig2:b} one can see how the ration depends on value of $\omega$. As we have already mention the relative distance between two particles become larger when the $\omega$ decreases. The shape of the curve may be explained as follows: the total energy can be divided into several (not equal!) parts. This parts are: the term containing the potential energy provided by harmonic oscillator, the kinetic energy provided by harmonic oscillator and the potential energy coming from Column interaction, that do not depend on the $\omega$ at all. At some point, when $\omega$ become small, the potential part dominates the kinetic part. This is actually the effect we see from the plot. If we go even further and compute the energy and distance for the case when $\omega =0$ we would get very large value for the distance and a very small energy (as soon as it only contain the potential Column inversely proportional to distance, that becomes very large).

\newpage
\clearpage
\section{Conclusion and further research}\label{conc}
In this project we studied two electrons in three-dimensional harmonic oscillator potential. The obtained results are in a good agreement with those obtained in project 2. In order to get even better result we need to construct more complicated trial wave function. This is not a simple trick to perform and will require much more time and knowledge compared with those we apply here. 

This project is a very first step into a field of science studying behavior of quantum-mechanical many-body systems. To apply used method to a system with larger number of particles object-orientation approach should be used. We used a very naive approach of minimizing local energy and we can mention that such approach can be used just for rough calculation of parameters. One should employ more advanced method to find energy minimum as for example Newton method or even conjugate gradient method which has better convergence. 

\newpage

\begin{thebibliography} {9}

\bibitem {proj2}
A. Kukharenka, A. Gribkovskaya,
\textit
{Schr\"{o}dinger's equation for two electrons in three dimensional harmonic oscillator potential
}
https://github.com/andrei-fys/fys4150/blob/master/Project\_2/report/Project\_2\_fys4150\_fall\_2016.tex (2016)


\bibitem{three}
M. Taut. 
\textit{Two electrons in an external oscillator potential: Particular analytic solutions of a Coulomb correlation problem}.
Phys. Rev. A, 11.1993.


\bibitem{four}
D. Loss, D. P. DiVincenzo
\textit{
Quantum computation with quantum dots
}
Phys. Rev. A 57, 120 – Published 1 January 1998.

\bibitem{five}
P. Patel
\textit
{The First Full-Color Display with Quantum Dots
}
MIT Technology Review, February 22, 2011.

\bibitem {med}
Yuri Volkov
\textit
{Quantum dots in nanomedicine: recent trends, advances and unresolved issues
}
Biochemical and Biophysical Research Communications, Volume 468, Issue 3, 18 December 2015, Pages 419–427

\bibitem{one} 
Morten Hjorth-Jensen. 
\textit{Computational Physics
}. 
Lecture Notes Fall 2015, August 2015.


\bibitem {wigner}
E.P. Wigner
\textit
{On the interaction of electrons in metals
}
Phys. Rev. B 46, 1002 (1934) 

\bibitem {Liboff}
Richard L. Liboff
\textit
{Introductory Quantum Mechanics
}
Pearson, 2007.


\end{thebibliography}

\end{document}
